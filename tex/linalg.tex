\documentclass{article} 
 \usepackage[fleqn]{amsmath}
 \usepackage{titlesec}
 \setlength\mathindent{0pt}
 \title{Declarative Matrix Language: Linear Algebra Project} 
 \author{Matthew Moltzau} 
 \date{\today} 
 \begin{document} 
 \maketitle 
 \begin{flushleft}

\section*{Introduction}

I implemented my own language for the creation, manipulation, and output of two-dimensional
matrices. It is called `dml', which is short for `Declarative Matrix Language'. For more
information on dml, check the pdf titled `Language Documentation of dml'.

%Project Description
%Program Syntax
%Current Implementation Progress
%Application Problems
%Challenges

\section*{Application 3.2}

Application 3.2 introduces automated row operations with Matlab, Mathematica, and Mable.
Using my program, I was able to represent the same row operations and get the same results.
The output has been re-formatted for this document.
\newline

Normally, there are some restrictions placed on row operations, including that a row
shouldn't be multiplied by 0. In my implementation there are no such constraints and
I tried to make it so that the rows could be visualized as a part of a flexible expression
instead of having to call a function followed by a list of parameters.
\newline


A ::=
\[
\begin{bmatrix}
1  & 2  & 1  & 4 \\
3  & 8  & 7  & 20 \\
2  & 7  & 9  & 23 \\
\end{bmatrix} 
\]

Performing row operations:   \newline \newline
$A[1] - 3 * A[0] \rightarrow A[1];$     \newline
$A[2] - 2 * A[0] \rightarrow A[2];$     \newline
$A[1] / 2 \rightarrow A[1];       $     \newline
$A[2] - 3 * A[1] \rightarrow A[2];$     \newline


Result of row operations:
\[ 
\begin{bmatrix}
1  & 2  & 1  & 4 \\
0  & 1  & 2  & 4 \\
0  & 0  & 1  & 3 \\
\end{bmatrix} 
 \]

Swapping is also a row operation, despite it not being used above. Using the previous form,
we can now swap A[0] and A[2] to allow forward-substitution instead of back-substitution.
\newline


Result of swapping rows:
\[ 
\begin{bmatrix}
0  & 0  & 1  & 3 \\
0  & 1  & 2  & 4 \\
1  & 2  & 1  & 4 \\
\end{bmatrix} 
 \]



%\sectionbreak
%\clearpage
\section*{Application 3.3}

After getting the basic row operations complete, row reduction becomes possible to implement.
In my language, row reduction works best with decimal values instead of integers (the integer
values become truncated). When I set up the problem in a script, I created an integer matrix
first and had to cast it to decimal. After the computation was complete I casted it back to
an integer matrix to get prettier output.
\newline

The first example is the same matrix from Application 3.2, except we are interested in the
fully-reduced row echelon form.
\newline

A ::=
\[ 
\begin{bmatrix} 
1  & 2  & 1  & 4 \\
3  & 8  & 7  & 20 \\
2  & 7  & 9  & 23 \\
\end{bmatrix} 
 \]
rref:
\[ 
\begin{bmatrix} 
1  & 0  & 0  & 5 \\
0  & 1  & 0  & -2 \\
0  & 0  & 1  & 3 \\
\end{bmatrix} 
 \]


There are six systems of equations given in the textbook. I checked to make sure they are
all correct. \newline


Matrix 1:
\[ 
\begin{bmatrix} 
17  & 42  & -36  & 213 \\
13  & 45  & -34  & 226 \\
12  & 47  & -35  & 197 \\
\end{bmatrix} 
 \]
rref:
\[ 
\begin{bmatrix} 
1  & 0  & 0  & 39 \\
0  & 1  & 0  & 27 \\
0  & 0  & 1  & 44 \\
\end{bmatrix} 
 \]


Matrix 2:
\[ 
\begin{bmatrix} 
32  & 57  & -41  & 713 \\
23  & 43  & -37  & 130 \\
42  & -61  & 39  & 221 \\
\end{bmatrix} 
 \]
rref:
\[ 
\begin{bmatrix} 
1  & 0  & 0  & 17 \\
0  & 1  & 0  & 49 \\
0  & 0  & 1  & 64 \\
\end{bmatrix} 
 \]


Matrix 3:
\[ 
\begin{bmatrix} 
231  & 157  & -241  & 420 \\
323  & 181  & -376  & 412 \\
542  & 161  & -759  & 419 \\
\end{bmatrix} 
 \]
rref:
\[ 
\begin{bmatrix} 
1.000000  & -0.000000  & -0.000000  & 38.358214 \\
0.000000  & 1.000000  & 0.000000  & -18.629199 \\
-0.000000  & -0.000000  & 1.000000  & 22.887814 \\
\end{bmatrix} 
 \]


Matrix 4:
\[ 
\begin{bmatrix} 
837  & 667  & -729  & 1659 \\
152  & -179  & -975  & 1630 \\
542  & 328  & -759  & 1645 \\
\end{bmatrix} 
 \]
rref:
\[ 
\begin{bmatrix} 
1  & 0  & 0  & 77 \\
0  & 1  & 0  & -69 \\
0  & 0  & 1  & 23 \\
\end{bmatrix} 
 \]


Matrix 5:
\[ 
\begin{bmatrix} 
49  & -57  & 37  & -59  & 97 \\
73  & -15  & -19  & -22  & 99 \\
52  & -51  & 14  & -29  & 89 \\
13  & -27  & 27  & -25  & 73 \\
\end{bmatrix} 
 \]
rref:
\[ 
\begin{bmatrix} 
1  & 0  & 0  & 0  & 5 \\
0  & 1  & 0  & 0  & 3 \\
0  & 0  & 1  & 0  & 7 \\
0  & 0  & 0  & 1  & 4 \\
\end{bmatrix} 
 \]


Matrix 6:
\[ 
\begin{bmatrix} 
64  & -57  & 97  & -67  & 485 \\
92  & 77  & -34  & -37  & 486 \\
44  & -34  & 53  & -34  & 465 \\
27  & 57  & -69  & 29  & 464 \\
\end{bmatrix} 
 \]
rref:
\[ 
\begin{bmatrix} 
1  & 0  & 0  & 0  & 17 \\
0  & 1  & 0  & 0  & 23 \\
0  & 0  & 1  & 0  & 37 \\
0  & 0  & 0  & 1  & 43 \\
\end{bmatrix} 
 \]


\section*{Application 3.5}
I was not able to get to Application 3.5.


\end{flushleft}
\end{document}

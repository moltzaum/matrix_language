\batchmode
%This Latex file is machine-generated by the BNF-converter

\documentclass[a4paper,11pt]{article}
\setlength{\parindent}{0mm}
\setlength{\parskip}{1mm}
\usepackage{hyperref}
\title{Language Documentation of dml}
\author{Created by Matthew Moltzau, with the help of BNFC}

\begin{document}
\maketitle

\newcommand{\emptyP}{\mbox{$\epsilon$}}
\newcommand{\terminal}[1]{\mbox{{\texttt {#1}}}}
\newcommand{\nonterminal}[1]{\mbox{$\langle \mbox{{\sl #1 }} \! \rangle$}}
\newcommand{\arrow}{\mbox{::=}}
\newcommand{\delimit}{\mbox{$|$}}
\newcommand{\reserved}[1]{\mbox{{\texttt {#1}}}}
\newcommand{\literal}[1]{\mbox{{\texttt {#1}}}}
\newcommand{\symb}[1]{\mbox{{\texttt {#1}}}}

\section*{Introduction}

`dml' stands for `declarative matrix language' and is meant to be a tool for operations
on 2D matrices. The specification is somewhat incomplete since many matrix operations
have not been included so far.

This project was only possible thanks to \href{http://bnfc.digitalgrammars.com}{BNFC}, which
generated many useful files including a lexer, parser, and abstract syntax tree.

\subsection*{Language Features}

\begin{itemize}
    
    \item[] \textbf{Variables} \newline
        Integers, doubles, and matrices are the only types in dml. All types are interpreted
        in a manner similar to python, making dml an interpreted language.
    
    \item[] \textbf{Mathematical Operators} \newline
        These include + - / * on both doubles and integer types. Addition and subtraction
        may be performed on rows, where a row is a subscripted matrix like so: A[i].
        
        Rows have division and multiplication with scalars, and even have a swap operator
        `$<$$-$$>$', which means all standard row operations can be represented.
        
    \item[] \textbf{Output} \newline
        Standard output is accessed via using the `$=>$' operator on `out', while
        using `tex' instead generates output in a \LaTeX\ file. You will likely need to still
        edit any tex output, but it can at least automate creating matrices in \LaTex.
        
    \item[] \textbf{ref and rref} \newline
        The only implemented matrix operations as of now are ref and rref, which build on
        some of the row operations. Reduced-row echelon format doesn't work well with integers,
        so I also implemented casting. Casting allows rref to work with doubles, then cast back
        to an integer, however, matrix casting can only be done explicitly.
    
    \item[] \textbf{Comments} \newline
        Single-line comments begin with {\symb{//}}. \\Multiple-line comments are  enclosed with {\symb{/*}} and {\symb{*/}}.
    
\end{itemize}

\subsection*{Implementation Challenges}
One of the particularly difficult features to implement was allowing a matrix to hold both
doubles and integers. c doesn't have a strong support for generic types, and I would have
to code some sections differently, depending on the type being used.

\section*{The Syntactic Structure of dml}

Non-terminals are enclosed between $\langle$ and $\rangle$. 
The symbols  {\arrow}  (production),  {\delimit}  (union) 
and {\emptyP} (empty rule) belong to the BNF notation. 
All other symbols are terminals.\\

\begin{tabular}{lll}
{\nonterminal{Program}} & {\arrow}  &{\nonterminal{ListStm}}  \\
\end{tabular}\\

\begin{tabular}{lll}
{\nonterminal{Stm}} & {\arrow}  &{\terminal{create}} {\nonterminal{String}} {\terminal{titled}} {\nonterminal{String}} {\terminal{for}} {\nonterminal{String}}  \\
 & {\delimit}  &{\nonterminal{Ident}} {\terminal{{$=$}{$>$}}} {\terminal{tex}}  \\
 & {\delimit}  &{\nonterminal{String}} {\terminal{{$=$}{$>$}}} {\terminal{tex}}  \\
 & {\delimit}  &{\terminal{close}} {\terminal{tex}}  \\
 & {\delimit}  &{\terminal{out}} {\terminal{{$=$}{$>$}}} {\terminal{tex}}  \\
 & {\delimit}  &{\nonterminal{Ident}} {\terminal{{$=$}{$>$}}} {\terminal{out}}  \\
 & {\delimit}  &{\nonterminal{String}} {\terminal{{$=$}{$>$}}} {\terminal{out}}  \\
 & {\delimit}  &{\nonterminal{Row}} {\terminal{{$=$}{$>$}}} {\terminal{out}}  \\
 & {\delimit}  &{\nonterminal{Ident}} {\terminal{::{$=$}}} {\nonterminal{ListRow}}  \\
 & {\delimit}  &{\nonterminal{Row}} {\terminal{{$<$}{$-$}{$>$}}} {\nonterminal{Row}}  \\
 & {\delimit}  &{\nonterminal{Row}} {\terminal{{$-$}{$>$}}} {\nonterminal{Row}}  \\
 & {\delimit}  &{\nonterminal{Row}} {\terminal{{$=$}}} {\nonterminal{Row}}  \\
 & {\delimit}  &{\nonterminal{Ident}} {\terminal{{$=$}{$>$}}} {\terminal{rref}}  \\
 & {\delimit}  &{\nonterminal{Ident}} {\terminal{{$=$}{$>$}}} {\terminal{ref}}  \\
 & {\delimit}  &{\nonterminal{Ident}} {\terminal{{$=$}}} {\nonterminal{Exp}}  \\
 & {\delimit}  &{\nonterminal{Ident}} {\terminal{{$=$}{$>$}}} {\terminal{double}}  \\
 & {\delimit}  &{\nonterminal{Ident}} {\terminal{{$=$}{$>$}}} {\terminal{int}}  \\
 & {\delimit}  &{\terminal{pause}}  \\
\end{tabular}\\

\begin{tabular}{lll}
{\nonterminal{Row}} & {\arrow}  &{\terminal{[}} {\nonterminal{ListExp}} {\terminal{]}}  \\
 & {\delimit}  &{\nonterminal{Row}} {\terminal{{$+$}}} {\nonterminal{Row1}}  \\
 & {\delimit}  &{\nonterminal{Row}} {\terminal{{$-$}}} {\nonterminal{Row1}}  \\
 & {\delimit}  &{\nonterminal{Row1}}  \\
\end{tabular}\\

\begin{tabular}{lll}
{\nonterminal{Row1}} & {\arrow}  &{\nonterminal{Exp2}} {\terminal{*}} {\nonterminal{Row2}}  \\
 & {\delimit}  &{\nonterminal{Row2}} {\terminal{/}} {\nonterminal{Exp2}}  \\
 & {\delimit}  &{\nonterminal{Row1}} {\terminal{*}} {\nonterminal{Exp2}}  \\
 & {\delimit}  &{\nonterminal{Row2}}  \\
\end{tabular}\\

\begin{tabular}{lll}
{\nonterminal{Row2}} & {\arrow}  &{\nonterminal{Ident}} {\terminal{[}} {\nonterminal{Exp}} {\terminal{]}}  \\
 & {\delimit}  &{\terminal{(}} {\nonterminal{Row}} {\terminal{)}}  \\
\end{tabular}\\

\begin{tabular}{lll}
{\nonterminal{Exp}} & {\arrow}  &{\nonterminal{Exp}} {\terminal{{$+$}}} {\nonterminal{Exp1}}  \\
 & {\delimit}  &{\nonterminal{Exp}} {\terminal{{$-$}}} {\nonterminal{Exp1}}  \\
 & {\delimit}  &{\nonterminal{Exp1}}  \\
\end{tabular}\\

\begin{tabular}{lll}
{\nonterminal{Exp1}} & {\arrow}  &{\nonterminal{Exp1}} {\terminal{*}} {\nonterminal{Exp2}}  \\
 & {\delimit}  &{\nonterminal{Exp1}} {\terminal{/}} {\nonterminal{Exp2}}  \\
 & {\delimit}  &{\nonterminal{Exp2}}  \\
\end{tabular}\\

\begin{tabular}{lll}
{\nonterminal{Exp2}} & {\arrow}  &{\nonterminal{SInteger}}  \\
 & {\delimit}  &{\nonterminal{Double}}  \\
 & {\delimit}  &{\nonterminal{Ident}}  \\
 & {\delimit}  &{\terminal{e}}  \\
 & {\delimit}  &{\terminal{pi}}  \\
 & {\delimit}  &{\terminal{(}} {\nonterminal{Exp}} {\terminal{)}}  \\
\end{tabular}\\

\end{document}

